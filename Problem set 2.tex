\documentclass[12pt]{article}%
\usepackage[paper=portrait,pagesize]{typearea}
\usepackage{amssymb}
\usepackage{amsfonts}
\usepackage{amsmath}
\usepackage{hyperref}
\usepackage{lscape}
\usepackage{comment}
\usepackage[flushleft]{threeparttable}
\usepackage{float}
\usepackage[nohead]{geometry}
\usepackage[singlespacing]{setspace}
\usepackage[paper=portrait,pagesize]{typearea}
\usepackage{amssymb}
\usepackage{amsfonts}
\usepackage{multicol}
\usepackage{amsmath}
\usepackage{hyperref}
\usepackage[nameinlink,noabbrev]{cleveref}
\usepackage{lscape}
\usepackage{float}
\usepackage[nohead]{geometry}
\usepackage[singlespacing]{setspace}
\usepackage[bottom]{footmisc}
\usepackage{indentfirst}
\usepackage{endnotes}
\usepackage{graphicx}%
\usepackage{afterpage}
\usepackage{subfig}
\usepackage{rotating}
\newcommand\tab[1][1cm]{\hspace*{#1}}
\DeclareMathOperator*{\Max}{Max}
\newcommand\numberthis{\addtocounter{equation}{1}\tag{\theequation}}
\def\dotfill#1{\cleaders\hbox to #1{.}\hfill}
\newcommand\dotline[2][.5em]{\leavevmode\hbox to #2{\dotfill{#1}\hfil}}
%\usepackage[backend=biber,style=alphabetic,sorting=ynt]{biblatex}
%\addbibresource{bibliocopulas.bib}
\usepackage[round,sort,comma,authoryear]{natbib}
\setcounter{MaxMatrixCols}{30}
\newtheorem{theorem}{Theorem}
\newtheorem{acknowledgement}{Acknowledgement}
\newtheorem{algorithm}[theorem]{Algorithm}
\newtheorem{axiom}[theorem]{Axiom}
\newtheorem{case}[theorem]{Case}
\newtheorem{claim}[theorem]{Claim}
\newtheorem{conclusion}[theorem]{Conclusion}
\newtheorem{condition}[theorem]{Condition}
\newtheorem{conjecture}[theorem]{Conjecture}
\newtheorem{corollary}[theorem]{Corollary}
\newtheorem{criterion}[theorem]{Criterion}
\newtheorem{definition}[theorem]{Definition}
\newtheorem{example}[theorem]{Example}
\newtheorem{exercise}[theorem]{Exercise}
\newtheorem{lemma}[theorem]{Lemma}
\newtheorem{notation}[theorem]{Notation}
\newtheorem{problem}[theorem]{Problem}
\newtheorem{proposition}{Proposition}
\newtheorem{remark}[theorem]{Remark}
\newtheorem{solution}[theorem]{Solution}
\newtheorem{summary}[theorem]{Summary}
\newenvironment{proof}[1][Proof]{\noindent\textbf{#1.} }{\ \rule{0.5em}{0.5em}}
\newcommand{\pd}[2]{\frac{\partial#1}{\partial#2}}
\makeatletter
\def\@biblabel#1{\hspace*{-\labelsep}}
\makeatother
\geometry{left=1in,right=1in,top=1.00in,bottom=1.0in}
%\renewcommand*\abstractname{Summary}

\begin{document}

\title{Fall 2019 - ECON 634 - Advance Macroeconomics - Problem Set 2}
\author{Elisa Taveras Pena\footnote{E-mail address: \href{mailto:etavera2@binghamton.edu}{etavera2@binghamton.edu}  }\\
Binghamton University}
\maketitle

\sloppy%avoids the breakage of words at the end of lines, by adjusting spaces between words inside the lines

\onehalfspacing

\begin{enumerate}
	\item 
	
	Since the Resource constraint (Social Planner Problem) is $c_t=A_tk_{t}^\alpha+(1-\delta)k_{t}-k_{t+1}$ we can write the budget constraint  in recursive form as  $c=Ak^\alpha+(1-\delta)k-k'$ 
	
	\tab \textbf{ $\bullet$ State variable}: $k,A$ 
	
	\tab \textbf{ $\bullet$ Control variable}: $k'$
	
	
	
	Therefore, the Bellman equation:
	
	\begin{align*}
	&V(k,A)=\Max_{k'} \left\lbrace \frac{\left( Ak^\alpha+(1-\delta)k-k'\right) ^{1-\sigma}}{1-\sigma}+ \beta \sum_{A' \in A} \Pi(A'|A)V(k',A')\right\rbrace 
	& \notag \\
	&\text{subject to} \notag \\
	& \tab c \in [0,f(k)] \numberthis\\
	& \tab k' \in [0,f(k)] \ \numberthis\\
	\end{align*} 
	
	\item Using the VFI, the Graphs are like follows:
	
	\begin{center}
		\includegraphics[width=1\linewidth]{VF}
	\end{center}
	
	
	As we can see, both are increasing and concave functions. The function is increasing in A. 
	
	\item  The Policy function over $k$ looks as follows:
	
	\begin{center}
		\includegraphics[width=1\linewidth]{g_k}
	\end{center}

This relationship is increasing in $k$  and  $A$.
	
	Assuming that by saving, it means what is left from production after consumption: $s=k^{\prime}-(1-\delta)k$, the Saving over $k$ looks as follows:
	
	\begin{center}
		\includegraphics[width=1\linewidth]{saving}
	\end{center}
	
	where this relationship is increasing in A. In term of $k$, which is more obvious for $A_l$, it is increasing up t a point and then decrease over the periods. 

	
	\item Using the results, I get a $sd(y)= 0.1412$, which does not match the results from the data. 
	
		 Ideally, I should do a while loop and try for a low tolerance between my simulated standard deviation and the actual standard deviation. I decide against it and just try randomly picking numbers for the High value different to $A_h=1.1$ and backing out the $A_l$ values.	 	
	
	Doing the markov change process to find stationary probabilities:
	
		\[
	\begin{pmatrix}\bar{\pi}_{h} & \bar{\pi}_{l}\end{pmatrix}=\begin{pmatrix}\bar{\pi}_{h} & \bar{\pi}_{l}\end{pmatrix}P
	\]
	
	This means
	
	\[
	\begin{pmatrix}\bar{\pi}_{h} & \bar{\pi}_{l}\end{pmatrix}=\begin{pmatrix}\bar{\pi}_{h} & \bar{\pi}_{l}\end{pmatrix}\left[\begin{array}{cc}
	0.977 & 0.023\\
	0.074 & 0.926
	\end{array}\right],
	\]
	
			\[
	\begin{pmatrix}\bar{\pi}_{h} & \bar{\pi}_{l}\end{pmatrix}=\begin{pmatrix} 0.977\bar{\pi}_{h}+ 0.074\bar{\pi}_{l} & 	0.023\bar{\pi}_{h}+0.926\bar{\pi}_{l}\end{pmatrix},
	\]
	
		Therefore,
	
	\begin{align*}
	&\bar{\pi}_{h}=  3.22 \bar{\pi}_{l}  \numberthis \\
	&\bar{\pi}_{l}=  0.31 \bar{\pi}_{h} \numberthis \\	
	\end{align*}
	
		Since $\bar{\pi}_{h}+\bar{\pi}_{l}=1$
	
	\begin{align*}
	&4.22\bar{\pi}_{l}= 1 \numberthis \\
	&\bar{\pi}_{l}= 0.24  \numberthis \\
	&\bar{\pi}_{h}=  0.76   \numberthis \\	
	\end{align*}
	
	 Therefore, I will try first, $A_h=1.001$, the standard deviation is $sd(y)=0.4888$, which means I need to keep trying for a smaller value of $A_h$: 
	 
	 \begin{itemize}
	 		\item Using $A_h=1.01$, then  $sd(y)=0.1541$
	 	\item Using $A_h=1.001$, then  $sd(y)=0.1469$
	 	\item Using $A_h=1.0001$, then  $sd(y)=0.1467$
	 	\item Using $A_h=1.0000001$, then  $sd(y)=0.1467$
	 \end{itemize}
 
 And it stays there. Obviously, this compare with the information we saw in class, the numbers are different and quite high and very difficult to find the standard deviation that match the data. Since anything smaller than this value will be consider the two possible values to be equal to 1, I exhaust my search here. 
	
	\item See Code \textbf{VFIP5}. Using two loops over all $K$ is quite slow and does spend a long time to find a solution. 
	
	The time on the previous program \textbf{Elapsed time is 15.301668 seconds.}
	For the second one, limiting to the  \textbf{Elapsed time is 1049.453223 seconds.} Therefore, time is significantly slower for this one. 
	
 I get differents results, however, the VFI looks similar: 
	
	\begin{center}
		\includegraphics[width=1\linewidth]{VFP5}
	\end{center}
	
\end{enumerate}

\strut

\onehalfspacing

\end{document}
